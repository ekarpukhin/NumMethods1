\documentclass[a4paper,12pt]{article}
\usepackage[T2A]{fontenc}
\usepackage[utf8]{inputenc}
\usepackage[english, russian]{babel}
\usepackage{minted}
\usepackage{verbatim}
\usepackage{graphicx}
\usepackage{caption}
\usepackage{subcaption}
\usepackage{listings}

\usepackage[left=2.5cm,right=2.5cm,top=2cm,bottom=2cm]{geometry}

\newenvironment{longlisting}{\captionsetup{type=listing}}{}


\begin{document}

\tableofcontents

\section{Расчет частичных сумм ряда}
\subsection{Формулировка задачи}
Найти сумму ряда аналитически. Вычислить значения частичных сумм ряда 
и найти величину погрешности для значений $N = 10, 10^2, 10^3, 10^4, 10^5$
   $$S(N) = \sum_0^N a_n$$
   $$a_n = \frac{48}{5(n^2 +6n + 8)}$$
   
\subsection{Аналитический расчет суммы ряда}
Выведм формулу суммы ряда:
   $$S = \sum_{0}^{\infty} \frac{48}{5(n^2 +6n + 8)}$$
Упростим член ряда:
   $$a_n =  \frac{48}{5(n^2 +6n + 8)} = \frac{24}{5}(\frac{1}{n+2} - \frac{1}{n+4})$$
Тогда сумма упростится:
   $$S = \sum_{0}^{\infty}  = \frac{24}{5}(\frac{1}{n+2} - \frac{1}{n+4}) = \sum_{0}^{\infty}  = \frac{24}{5}(\frac{1}{2} + \frac{1}{3} - \frac{1}{N+3} - \frac{1}{N+4})$$

Перейдем к пределу:
   $$S = \lim_{s\rightarrow \infty}S_N = \frac{12}{5} + \frac{8}{5} = 4$$

\subsection{Код на Python и результаты}

\begin{longlisting}
\inputminted{python}{1_4.py}
\end{longlisting}


\section{Квадратное уравнение}
\subsection{Формулировка задачи}
Дано квадратное уравнение. Предполагается, что один из коэффициентов уравнения (помечен $*$) получен в результате округления. Произвести теоретическую оценку погрешностей корней в зависимости от погрешности коэффициента. Вычислить корни уравнения при нескольких различных значениях коэффициента в пределах заданной точности, сравнить.
$$x^2+bx+c = 0$$
$$b^* = -39,6$$
$$c = -716,85 $$
    
\subsection{Оценка погрешности}

$$f(b) = x_{1,2} = \frac{-b \pm \sqrt{b^2-4ac}}{2a}$$

Оценка:
$$\Delta f = f'(b)\Delta b$$
$$f'(b) = \frac{1}{2}(\frac{b}{\sqrt{b^2-4c}}-1)$$



\section{Машинная точность}
\subsection{Формулировка задачи}
\subsection{Код на Python}

\begin{longlisting}
\inputminted{python}{6.py}
\end{longlisting}

\subsection{Код на C++}

\begin{longlisting}
\inputminted{c++}{6.cpp}
\end{longlisting}

\section{Компланарность}

\begin{longlisting}
\inputminted{python}{1_10.py}
\end{longlisting}

\end{document}